\documentclass[review]{elsarticle}

%BaBar symbols
\include {babarsym}
% symbols unique to this paper
\def\deltaray{\ensuremath{{\rm \delta}-ray}\xspace}
\def\deltarays{\ensuremath{{\rm \delta}-rays}\xspace}
\def\eetomm{\ensuremath{ \epem \to \mumu}\xspace}
\def\mutoe{\ensuremath{\mu \to e}\xspace}


\usepackage{lineno,hyperref}
\modulolinenumbers[5]

\journal{\nimBaseA}

%%%%%%%%%%%%%%%%%%%%%%%
%% Elsevier bibliography styles
%%%%%%%%%%%%%%%%%%%%%%%
%% To change the style, put a % in front of the second line of the current style and
%% remove the % from the second line of the style you would like to use.
%%%%%%%%%%%%%%%%%%%%%%%

%% Numbered
%\bibliographystyle{model1-num-names}

%% Numbered without titles
%\bibliographystyle{model1a-num-names}

%% Harvard
%\bibliographystyle{model2-names.bst}\biboptions{authoryear}

%% Vancouver numbered
%\usepackage{numcompress}\bibliographystyle{model3-num-names}

%% Vancouver name/year
%\usepackage{numcompress}\bibliographystyle{model4-names}\biboptions{authoryear}

%% APA style
%\bibliographystyle{model5-names}\biboptions{authoryear}

%% AMA style
%\usepackage{numcompress}\bibliographystyle{model6-num-names}

%% `Elsevier LaTeX' style
\bibliographystyle{elsarticle-num}
%%%%%%%%%%%%%%%%%%%%%%%

\begin{document}

\begin{frontmatter}

\title{High Energy \deltaray Production by Muons in the \lbabar \xspace Detector}
%\tnoteref{mytitlenote}}
%\tnotetext[mytitlenote]{Fully documented templates are available in the elsarticle package on \href{http://www.ctan.org/tex-archive/macros/latex/contrib/elsarticle}{CTAN}.}

%% Group authors per affiliation:
\author[LBLaddress]{David N. Brown\corref{mycorrespondingauthor}}
\author[LBLaddress]{Brian Chan}
\author[LBLaddress]{Tom-Eric Haugen}
\address[LBLaddress]{Lawrence Berkeley National Lab}

\begin{abstract}
In this note we present a measurement of the \deltaray production cross section
by muons in the \babar detector \cite{babarnim}, for \deltarays between 50 and 300 MeV.
The \babar experiment
recorded over 500M \eetomm events near the upsilon resonances over its 7 year
run, producing over 1G muons with a lab frame energies between 3 and 9 GeV.
These muons produced \deltarays
when traversing the beampipe and other material in the BaBar inner tracker
region.  We identify
\deltaray candidates by associating a reconstructed low-momentum negatively
charged track with a nearby reconstructed high-momentum track in events
consistent with the \eetomm process, where
the two track’s point of closest approach (POCA) is consistent with a known
piece of dense material.  A detailed simulation predicts the primary \deltaray
background comes from electrons produced in the
conversion of photons radiated by the muons, which we measure by applying the
same reconstruction algorithm to positively charged tracks.  Monte Carlo
simulation is used to estimate the remaining backgrounds,
which are estimated to be less than 1\% of the predicted signal.  Simulation is
also used to estimate the reconstruction efficiency, the \deltaray energy
resolution, and to predict the mis-identification of the radiator material.  We
extract a cross-section by normalizing the corrected production rate to the
estimated material traversed by the associated reconstructed muon candidate
track, as a function of the reconstructed electron candidate track energy.  The
cross-section errors are dominated by uncertainty in the radiator material
mass, which is roughly 5\% and coherent across the energy points measured.  The
resulting cross-section is found to be consistent with theoretical predictions
within errors.

\end{abstract}

\begin{keyword}
\texttt{Material Interactions}
\end{keyword}

\end{frontmatter}

\linenumbers

\section{Introduction}
Atomic electrons ejected by high-energy charged particles passing through
matter (\deltarays) can be a background for experiments using electrons as
signals.  An example is the \mutoe experiment \cite{mutoe}, which will search for muon
to electron conversion in the field of an aluminum nucleus.  In that experiment
the signal signature is the observation of an isolated 105 MeV electron.
Simulations of the \mutoe experiment based on the Geant \cite{g4} Monte Carlo
indicate \deltarays produced by cosmic ray muons mimic the signal process
at a rate many times the experimental sensitivity goal, motivating
the need for an active cosmic ray veto.  Validating the \mutoe experiment design
and estimating the resulting experimental sensitivity thus depends crucially on
understanding the \deltaray production cross section.

A quantitative theoretical explanation of \deltaray production was derived in
the 1930s from quantum electrodynamics \cite{drtheory}, and is incorporated in the
\deltaray production model used by the Geant Monte Carlo \cite{g4}.  This theory
and has been shown to agree with measurements for low energy ($< 1$ MeV)
electrons \cite{drexpt}, however it has not been verified experimentally for
\deltarays in the energy range relevant to the \mutoe experiment.


\section{The \babar Detector and Data Sample}

The \babar experiment operated for 7 years at the 
\pepII collider at SLAC \cite{pepII}.
This provided approximately ** accumulated luminosity at the \FourS.  Additional
off-resonance samples and samples at other $\Upsilon$ resonances were recorded but
not used in this analysis.  The process \eetomm produced a sample of over
$10^9$ muons between roughly 3 and 9 GeV.

The \babar detector was designed to record the results of $epem$ collisions at
a center of mass energy near the Upsilon resonances.  The \babar
beampipe, Silicon Vertex Tracker (SVT) layers , Carbon fiber Support Tube
(CST), and  Drift Chamber (DCH) inner wall each presented a few  \% of a
radiation length of thin material to normal incident muons, providing radiators
for \deltaray production.  A detailed description of the material composition
of these structures is given in table \ref{materials}, while more general
information about the \babar detector can be found in \cite{babarnim}.

Charged particles in \babar produced signals in the SVT and DCH, which were
reconstructed as tracks using standard pattern recognition algorithms.  High
transverse momentum ($p_t > 100 MeV/c$) were identified primarily in the DCH,
while low transverse momentum tracks were found primarily in the SVT.  Tracks
identified in the DCH (SVT) were extrapolated into the SVT (DCH) to produce
complete tracks.  Final tracks were fit with a Kalman filter that compensated
for the effects of the energy lost in traversed materials and the inhomogeneous
\babar magnetic field.  Electrons with transverse momentum above 50 MeV/c (100
MeV/c) from the \epem interaction point were reconstructed with $>$ 50%
efficiency by the SVT (DCH), respectively.  Details of the \babar track
reconstruction algorithms and its performance can be found in \cite{babarnim2}.

\section{Monte Carlo Simulation}

To simulate the \eetomm process we use the {\em KK2F} \cite{kk2f} generator, which
includes the effects of initial and final state radiation (*** more is needed
here).  The initial electron and positron 4-momenta are selected randomly
according to the distributions observed in the \babar \Y4S runs, as described
in \ref{lumipaper}.  Figure \ref{mumom} plots the lab-frame energy of muons
from  \eetomm predicted by the kk2f generator.  Only events in which both muons
are produced in the active tracking volume are included.  Overlaid on figure
\ref{mumonm} is the expected momentum spectrum of cosmic ray muons that
generate signals in the \mutoe tracker, as predicted
by the Daya Bay experiment (need a real reference here**)
\cite{dayabay} generator and the \mutoe simulation \cite{mutoe}.  A reasonable
overlap is seen.

The interaction of particles with the \babar detector is simulated using a
detailed model built using the Geant4 toolkit \cite{g4}.  Detector components
are modeled assuming the ideal (design) geometry with sim\section{The \babar
Detector and Data Sample}

The \babar experiment operated for 7 years at the SLAC B factory, which
provided approximately ** accumulated luminosity at the \Y4S.  Additional
off-resonance samples and samples at other Upsilon resonances were recorded but
not used in this analysis.  The process \eetomm produced a sample of over
$10^9$ muons between roughly 3 and 9 GeV.

The \babar detector was designed to record the results of $epem$ collisions at
a center of mass energy near the Upsilon resonances [babar].  The \babar
beampipe, {\em Silicon Vertex Tracker} (SVT) layers , {\em Carbon Fiber Support Tube}
(CST), and  {\em Drift Chamber} (DCH) inner wall each presented a few  \% of a
radiation length of thin material to normal incident muons, providing radiators
for \deltaray production.  A detailed description of the material composition
of these structures is given in table \ref{materials}, while more general
information about the \babar detector can be found in \cite{babarnim}.

Charged particles in \babar produced signals in the SVT and DCH, which were
reconstructed as tracks using standard pattern recognition algorithms.  High
transverse momentum ($p_t > 100 MeV/c$) were identified primarily in the DCH,
while low transverse momentum tracks were found primarily in the SVT.  Tracks
identified in the DCH (SVT) were extrapolated into the SVT (DCH) to produce
complete tracks.  Final tracks were fit with a Kalman filter that compensated
for the effects of the energy lost in traversed materials and the inhomogeneous
\babar magnetic field.  Electrons with transverse momentum above 50 MeV/c (100
MeV/c) from the $epem$ interaction point were reconstructed with $>$ 50%
efficiency by the SVT (DCH), respectively.  Details of the \babar track
reconstruction algorithms and its performance can be found in \cite{babarnim2}.

plified shapes, using materials mixtures with relative elemental composition
and masses set to agree with measurements made during construction, averaged
over the relevant volumes.  For instance, the beampipe in the region of the
$epem$ interaction point is modeled as a cylinder of $2.5cm**$ radius with $2mm***$
wall thickness, as a material mix composed mostly of Beryllium and water, along
with trace elements from platings and coatings, as listed in table
\ref{tab:materials}.  Detailed features such as the interior vanes used to
direct the cooling water flow were not modeled.  The material mix was
determined by integrating the volume of each components and dividing by the
total volume.  While the dimensions of the Beryllium components were well
measured during detector construction, the coating thicknesses varied
spatially, introducing some uncertainty.  In addition, the exact composition of
one of the coatings \cite{paint} was proprietary to the corporation which
produced it, and our values were based on oral communication.  Similarly, the
SVT wafers are modeled as parallelepipeds with dimensions set to the average of
those measured during construction \cite{babarnim}.  The material composition of
the flex circuits was estimated by dissolving a spare circuit in acid, and
precipitating out and weighing the metals.

The Geant4 \deltaray production is simulated based on … (more here).  Figure
\ref{draycomp} plots the \deltaray spectrum produced in a simple Geant4 Monte
Carlo simulation of a mono-energetic 4 GeV muon traversing $2 mm$ of pure
Beryllium.  Overlaid is a theoretical calculation based on \cite{drtheory},

showing good agreement in the range covered.  Note there are no free parameters
in this comparison.

The \babar detector response simulation was tuned to reproduce detailed models
of the low-level sensor response measurements, as described in \cite{babarnim2} chapter **.
Dead channels in the detector were monitored and recorded coherently as a
function of time.  Simulated data samples were produced suppressing a given set
of measured dead channels, with event counts proportional to the luminosity
recorded in the period in which that set was dead.

The same reconstruction algorithms were run on simulated data as on \babar
data.  The efficiency for reconstructing tracks in \babar data and simulation
has been found to agree within the measured uncertainties over the range of
momentum relevant to this measurement using well-defined test samples
\cite{trackeff}.

\section{\deltaray Reconstruction}



\section*{References}

\bibliography{drays}

\end{document}
