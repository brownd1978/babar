\section{Introduction}
Atomic electrons ejected by high-energy charged particles passing through
matter (\deltarays) can be a background for experiments using electrons as
signals.  An example is the \mutoe experiment \cite{mutoe}, which will search for muon
to electron conversion in the field of an aluminum nucleus.  In that experiment
the signal signature is the observation of an isolated 105 MeV electron.
Simulations of the \mutoe experiment based on the Geant \cite{g4} Monte Carlo
indicate \deltarays produced by cosmic ray muons mimic the signal process
at a rate many times the experimental sensitivity goal, motivating
the need for an active cosmic ray veto.  Validating the \mutoe experiment design
and estimating the resulting experimental sensitivity thus depends crucially on
understanding the \deltaray production cross section.

A quantitative theoretical explanation of \deltaray production was derived in
the 1930s from quantum electrodynamics \cite{drtheory}, and is incorporated in the
\deltaray production model used by the Geant Monte Carlo \cite{g4}.  This theory
and has been shown to agree with measurements for low energy ($< 1$ MeV)
electrons \cite{drexpt}, however it has not been verified experimentally for
\deltarays in the energy range relevant to the \mutoe experiment.

