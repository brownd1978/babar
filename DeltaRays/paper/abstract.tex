\begin{abstract}
In this note we present a measurement of the \deltaray production cross section
by muons in the \babar detector \cite{babarnim}, for \deltarays between 50 and 300 MeV.
The \babar experiment
recorded over 500M \eetomm events near the upsilon resonances over its 7 year
run, producing over 1G muons with a lab frame energies between 3 and 9 GeV.
These muons produced \deltarays
when traversing the beampipe and other material in the BaBar inner tracker
region.  We identify
\deltaray candidates by associating a reconstructed low-momentum negatively
charged track with a nearby reconstructed high-momentum track in events
consistent with the \eetomm process, where
the two track’s point of closest approach (POCA) is consistent with a known
piece of dense material.  A detailed simulation predicts the primary \deltaray
background comes from electrons produced in the
conversion of photons radiated by the muons, which we measure by applying the
same reconstruction algorithm to positively charged tracks.  Monte Carlo
simulation is used to estimate the remaining backgrounds,
which are estimated to be less than 1\% of the predicted signal.  Simulation is
also used to estimate the reconstruction efficiency, the \deltaray energy
resolution, and to predict the mis-identification of the radiator material.  We
extract a cross-section by normalizing the corrected production rate to the
estimated material traversed by the associated reconstructed muon candidate
track, as a function of the reconstructed electron candidate track energy.  The
cross-section errors are dominated by uncertainty in the radiator material
mass, which is roughly 5\% and coherent across the energy points measured.  The
resulting cross-section is found to be consistent with theoretical predictions
within errors.

\end{abstract}
