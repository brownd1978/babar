\section{The \babar Detector and Data Sample}

The \babar experiment operated for 7 years at the 
\pepII collider at SLAC \cite{pepII}.
This provided approximately ** accumulated luminosity at the \FourS.  Additional
off-resonance samples and samples at other $\Upsilon$ resonances were recorded but
not used in this analysis.  The process \eetomm produced a sample of over
$10^9$ muons between roughly 3 and 9 GeV.

The \babar detector was designed to record the results of $epem$ collisions at
a center of mass energy near the Upsilon resonances.  The \babar
beampipe, Silicon Vertex Tracker (SVT) layers , Carbon fiber Support Tube
(CST), and  Drift Chamber (DCH) inner wall each presented a few  \% of a
radiation length of thin material to normal incident muons, providing radiators
for \deltaray production.  A detailed description of the material composition
of these structures is given in table \ref{materials}, while more general
information about the \babar detector can be found in \cite{babarnim}.

Charged particles in \babar produced signals in the SVT and DCH, which were
reconstructed as tracks using standard pattern recognition algorithms.  High
transverse momentum ($p_t > 100 MeV/c$) were identified primarily in the DCH,
while low transverse momentum tracks were found primarily in the SVT.  Tracks
identified in the DCH (SVT) were extrapolated into the SVT (DCH) to produce
complete tracks.  Final tracks were fit with a Kalman filter that compensated
for the effects of the energy lost in traversed materials and the inhomogeneous
\babar magnetic field.  Electrons with transverse momentum above 50 MeV/c (100
MeV/c) from the \epem interaction point were reconstructed with $>$ 50%
efficiency by the SVT (DCH), respectively.  Details of the \babar track
reconstruction algorithms and its performance can be found in \cite{babarnim2}.
